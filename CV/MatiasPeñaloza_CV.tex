\documentclass[11pt,a4paper,sans]{moderncv}

% Estilo y colores del CV
\moderncvstyle{banking} % Estilo (casual, classic, banking, fancy)
\moderncvcolor{blue}    % Color principal (blue, orange, green, red, purple, grey)

% Configuración de la página
\usepackage[scale=0.75]{geometry} % Tamaño de los márgenes
\usepackage{color}

% Información personal
\name{Matias}{Peñaloza}
\title{Estudiante de Ingeniería Civil Informática}
\address{Quilpue, Region de Valparaiso}{Chile}{}
\phone[mobile]{+56 9 2092 2574}
\email{mpenaloza@usm.cl}
\homepage{madmti.github.io}
\social[linkedin]{matias-daniel-peñaloza-bustamante-26041733b}
\extrainfo{Estudiante de la USM}
\quote{Apasionado por el desarrollo de software, paginas web y la automatización.}

\begin{document}

% Encabezado
\makecvtitle

% Sección: Perfil
%%% Hablar de mi experiancia con las tecnologias.
\section{Perfil}
Soy estudiante de Ingeniería Civil Informática en la Universidad Federico Santa María (USM).
A lo largo de mi carrera, he estado explorando diversas tecnologías y lenguajes de programación con el fin de solucionar problemas reales y mejorar la eficiencia de procesos cotidianos.
He trabajado en proyectos personales que me permitieron entender las distintas etapas del desarrollo de software, desde la planificación hasta la implementación, despliegue e integracion continua.

La mejor parte del desarrollo de software es que siempre existen mejores y nuevas formas de hacer las cosas, lo que me motiva a seguir aprendiendo y mejorando mis habilidades.

% Sección: Educación
\section{Educación}
\cventry{2023--Presente}{Ingeniería Civil Informática}{Universidad Federico Santa María (USM)}{Valparaíso, Chile}{}{Actualmente en mi tercer año de estudios.}

% Sección: Habilidades técnicas
\section{Habilidades técnicas}
\cvitem{Lenguajes de programación}{Python, JavaScript, PHP, Java, Dart, C/C++}
\cvitem{Herramientas}{Git/GitHub, Docker, Bash, Visual Studio Code}
\cvitem{Frameworks}{Flutter, Node.js/Bun, Express, OpenAI, Vercel, Astro, React, Azure}
\cvitem{Bases de datos}{Cosmos DB, MongoDB, SQL}
\cvitem{Otros}{HTML, CSS, Sass, Tailwind, Django}

% Sección: Experiencia laboral
\section{Experiencia laboral}
\cventry{2022/Ene--2022/Nov}{Trabajador de construcción}{M y L servicios integrales}{Pichilemu, Región de O'Higgins}{Chile}{}

% Sección: Proyectos personales
%%% Poner los link a los proyectos en GitHub o en la web.
%%% Mas explicativo agregando quotes.
\section{Proyectos personales}
\cvitem{Aplicación de mapas interactivos}{Una aplicación para mostrar rutas a lugares de la universidad, utilizando el algortimo de Dijkstra y un mapeo de la universidad realizado con otra app unicamente creada para el desarrollo. Tecnologias utilizadas: Cosmos DB (Azure), Express y Flutter.}
\textbf{Repositorio:} \href{https://github.com/Room-Track}{\textcolor{blue}{Room-Track}}
\\\\
\cvitem{Aplicación web modular}{Una aplicación con sistema de plugins para gestionar y visualizar horarios, calcular notas y organizar actividades académicas. Se desarrollo una SDK para facilitar la creación de plugins y estandarizar su configuracion, los plugins son servidos desde una cdn y precompilados con Vite. Tecnologías utilizadas: Astro, Vercel, Turso, Tailwind.}
\textbf{Repositorio:} \href{https://github.com/madmti/Portal-now}{\textcolor{blue}{PORTAL-now}}

\textbf{Link:} \href{https://portal-now.vercel.app/}{\textcolor{blue}{portal-now.vercel.app}}
\\\\
\cvitem{Aplicación web interactiva}{Una aplicación web para incentivar el reciclaje en la comunidad USM a través de la interactividad, fueron mis primeros pasos en el desarrollo web. Tecnologias utilizadas: Django, SQLite y Sass.} % Django, SQLite y Sass.
\textbf{Repositorio:} \href{https://github.com/madmti/TechnoCat}{\textcolor{blue}{TechnoCat}}
\\\\
\cvitem{Portafolio web personal}{Una página web con Astro para mostrar proyectos personales.}
\textbf{Link:} \href{https://madmti.github.io/}{\textcolor{blue}{madmti.github.io}}

% Sección: Idiomas
\section{Idiomas}
\cvitemwithcomment{Español}{Nativo}{}
\cvitemwithcomment{Inglés}{Intermedio}{En proceso de mejora}

% Sección: Intereses
\section{Intereses}
\cvitem{Tecnología}{Desarrollo de aplicaciones web y móviles, automatización.}
\cvitem{Ciencia de datos}{Estadistica computacional, aprendizaje automático y explicabilidad (XAI).}
\cvitem{Otros}{Aprender nuevos lenguajes de programación y herramientas.}


%%% Voler a poner los links con descripción.
\section{Contacto y redes}
\cvitem{Teléfono}{\href{tel:+56920922574}{\textcolor{blue}{+56 9 2092 2574}}}
\cvitem{Correo}{\href{mailto:mpenaloza@usm.cl}{\textcolor{blue}{mpenaloza@usm.cl}}}
\cvitem{LinkedIn}{\href{https://www.linkedin.com/in/matias-daniel-peñaloza-bustamante-26041733b}{\textcolor{blue}{matias-daniel-peñaloza-bustamante}}}
\cvitem{GitHub}{\href{https://github.com/madmti}{\textcolor{blue}{github.com/madmti}}}
\cvitem{Portafolio}{\href{https://madmti.github.io}{\textcolor{blue}{madmti.github.io}}}

\end{document}
