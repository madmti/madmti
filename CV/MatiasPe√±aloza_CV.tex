\documentclass[11pt,a4paper,sans]{moderncv}

% Estilo y colores del CV
\moderncvstyle{banking} % Estilo (casual, classic, banking, fancy)
\moderncvcolor{blue}    % Color principal (blue, orange, green, red, purple, grey)

% Configuración de la página
\usepackage[scale=0.75]{geometry} % Tamaño de los márgenes
\usepackage{color}

% Información personal
\name{Matias}{Peñaloza}
\title{Estudiante de Ingeniería Civil Informática}
\address{Quilpue, Region de Valparaiso}{Chile}{}
\phone[mobile]{+56 9 2092 2574}
\email{mpenaloza@usm.cl}
\homepage{madmti.github.io}
\social[linkedin]{matias-daniel-peñaloza-bustamante-26041733b}
\extrainfo{Estudiante de la USM}
\quote{Apasionado por el desarrollo de software, paginas web y la automatización.}

\begin{document}

% Encabezado
\makecvtitle

% Sección: Perfil
%%% Hablar de mi experiancia con las tecnologias.
\section{Perfil}
Soy estudiante de Ingeniería Civil Informática en la Universidad Federico Santa María (USM).
A lo largo de mi carrera, he estado explorando diversas tecnologías y lenguajes de programación con el fin de solucionar problemas reales y mejorar la eficiencia de procesos cotidianos.
He trabajado en proyectos personales que me permitieron entender las distintas etapas del desarrollo de software, desde la planificación hasta la implementación, despliegue e integración continua.

% Sección: Expectativas
\section{Expectativas}
Estoy en busca de \textbf{realizar mi práctica} en una empresa que valore la innovación y el aprendizaje continuo, donde pueda aplicar mis conocimientos y seguir creciendo profesionalmente.
Sería ideal que la práctica pueda ser \textbf{remota} o \textbf{presencial en la V Región}. En caso de que la modalidad presencial sea fuera de la región, preferiría que la asistencia presencial represente solo un \textbf{20\%} del total de la práctica.


% Sección: Educación
\section{Educación}
\cventry{2023--Presente}{Ingeniería Civil Informática}{Universidad Federico Santa María (USM)}{Valparaíso, Chile}{}{Actualmente en mi tercer año de estudios.}

% Sección: Habilidades técnicas
\section{Habilidades técnicas}
\cvitem{Lenguajes de programación}{Python, JavaScript, PHP, Java, Dart, C/C++}
\cvitem{Herramientas}{Git/GitHub, Docker, Bash, Visual Studio Code}
\cvitem{Frameworks}{Flutter, Node.js/Bun, Express, OpenAI API, Vercel, Astro, React, SvelteKit, Azure}
\cvitem{Bases de datos}{Cosmos DB, MongoDB, SQL}
\cvitem{Otros}{HTML, CSS, Sass, Tailwind, Django}

% Sección: Experiencia laboral
\section{Experiencia laboral}
\cventry{2022/Ene--2022/Nov}{Trabajador de construcción}{M y L servicios integrales}{Pichilemu, Región de O'Higgins}{Chile}{}

% Sección: Proyectos personales
%%% Poner los link a los proyectos en GitHub o en la web.
%%% Mas explicativo agregando quotes.
\section{Proyectos personales}
\cvitem{Aplicación de mapas interactivos}{Una aplicación para mostrar rutas a lugares de la universidad, utilizando el algortimo de Dijkstra y un mapeo de la universidad realizado con otra app unicamente creada para el desarrollo. Tecnologias utilizadas: Cosmos DB (Azure), Express y Flutter.}
\textbf{Repositorio:} \href{https://github.com/Room-Track}{\textcolor{blue}{Room-Track}}
\\\\
\cvitem{Analisis de Datos en Automoviles}{Un analisis exploratorio de datos en un dataset de automoviles, que incluye limpieza de datos, transformación inversa, estimacion de parametros, Intervalos de Confianza, Prueba de bondad, Test de hipotesis y Regresion lineal. Utilizando Python y sus librerias de ciencia de datos.}
\textbf{Link (COLAB):} \href{https://colab.research.google.com/drive/1XRXwAjej0ckybiYi3LHFj-ltwO6ZsBH_?usp=sharing}{\textcolor{blue}{Analisis de Datos en Automoviles}}
\\\\
\cvitem{Aplicación Web SPA/SSG para Gestión Académica}{Una aplicación web moderna offline-first desarrollada como SPA con capacidades SSG usando SvelteKit. Permite gestionar horarios académicos, calcular notas con predicciones matemáticas (algoritmos de programacion lineal), organizar eventos y seguimiento del progreso universitario. Aprovecha localStorage, arquitectura de monorepo, y múltiples vistas interactivas (calendario, kanban, timeline). Tecnologías: SvelteKit, TypeScript, Bun, Tailwind CSS.}
\textbf{Link:} \href{https://ramolibre.vercel.app/}{\textcolor{blue}{ramolibre.vercel.app}}

\textbf{Repositorio:} \href{https://github.com/madmti/RamoLibre}{\textcolor{blue}{RamoLibre}}
\\\\
\cvitem{Portafolio web personal}{Una página web con Astro para mostrar proyectos personales.}
\textbf{Link:} \href{https://madmti.github.io/}{\textcolor{blue}{madmti.github.io}}

% Sección: Idiomas
\section{Idiomas}
\cvitemwithcomment{Español}{Nativo}{}
\cvitemwithcomment{Inglés}{Intermedio}{En proceso de mejora}

% Sección: Intereses
\section{Intereses}
\cvitem{Tecnología}{Desarrollo de aplicaciones web y móviles, automatización.}
\cvitem{Ciencia de datos}{Estadistica computacional, aprendizaje automático y explicabilidad (XAI).}
\cvitem{Otros}{Aprender nuevos lenguajes de programación y herramientas.}


%%% Voler a poner los links con descripción.
\section{Contacto y redes}
\cvitem{Teléfono}{\href{tel:+56920922574}{\textcolor{blue}{+56 9 2092 2574}}}
\cvitem{Correo}{\href{mailto:mpenaloza@usm.cl}{\textcolor{blue}{mpenaloza@usm.cl}}}
\cvitem{LinkedIn}{\href{https://www.linkedin.com/in/matias-daniel-peñaloza-bustamante-26041733b}{\textcolor{blue}{matias-daniel-peñaloza-bustamante}}}
\cvitem{GitHub}{\href{https://github.com/madmti}{\textcolor{blue}{github.com/madmti}}}
\cvitem{Portafolio}{\href{https://madmti.github.io}{\textcolor{blue}{madmti.github.io}}}

\end{document}
